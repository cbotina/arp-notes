\section{Domain-Specific Algorithms}
\label{sec:domain-specific-algorithms}

\textbf{Domain-specific algorithms} are problem-solving techniques where the agent designer encodes a known method for solving problems in a particular domain into a specialized algorithm.

\begin{figure}[H]
    \centering
    \includegraphics[width=0.8\textwidth]{img/domainhanoi.png}
    \caption{Example of a domain-specific algorithm: Recursive solution for the Towers of Hanoi problem. This algorithm is specifically designed for this puzzle and cannot be directly applied to other domains.}
    \label{fig:hanoi-algorithm}
\end{figure}

\subsection{How They Work}
\label{subsec:domain-algorithms-operation}

\begin{itemize}
    \item The designer identifies a solution method for a specific problem domain.
    \item This method is encoded into a specialized algorithm that can solve any problem instance within that domain.
    \item Flexibility can be improved by using \textbf{parameters} that configure the problem, allowing the same code to handle different problem instances within the domain.
\end{itemize}

\subsection{Limitations}
\label{subsec:domain-algorithms-limitations}

The main drawbacks of this approach are:

\begin{itemize}
    \item \textbf{Exhaustive anticipation required}: The designer must anticipate all possible scenarios the agent might encounter. In real-world environments, this is often too complex to achieve.
    
    \item \textbf{Domain restriction}: The algorithm only works for the specific domain it was designed for. It cannot be applied to problems outside that domain without significant modification.
\end{itemize}
