\section{Introduction to Decision Making}
\label{sec:decision-making}

Decision making is a complex process that requires evaluating and understanding environmental conditions. In many cases, decisions become so intricate that we need to exhaustively evaluate every possible scenario. This is where mathematics becomes useful in helping us make informed decisions.

When making decisions, the following elements are involved:

\begin{itemize}
    \item \textbf{Future effect}: What is the duration of the decision's impact? (short-term, long-term)
    \item \textbf{Reversibility}: Is it easy to reverse the decision?
    \item \textbf{Impact}: How many areas or domains are affected?
    \item \textbf{Quality}: Ethics, legality, behavioral principles, workplace relationships, etc.
    \item \textbf{Periodicity}: How frequently must this decision be made?
\end{itemize}

\subsection{Decision Classification}
\label{subsec:decision-classification}

Decisions can be classified into two main categories: \textbf{high-level} and \textbf{low-level} decisions. High-level decisions are strategic, have long-term consequences, are difficult to reverse, and affect multiple areas of an organization or system. They are typically exceptional and require careful consideration of various quality factors. In contrast, low-level decisions are operational, have minimal future impact, are easily reversible, and affect fewer areas. They are made frequently and have limited impact on important quality factors.

Table~\ref{tab:decision_classification} summarizes the characteristics that distinguish high-level from low-level decisions based on the elements involved in decision making.

\begin{table}[H]
\centering
\caption{Classification of decisions: High-level vs. Low-level}
\label{tab:decision_classification}
\begin{tabular}{p{4.5cm}p{4.5cm}p{4.5cm}}
\toprule
 & \textbf{High Level} & \textbf{Low Level} \\
\midrule
\textbf{Future Effect} & Affect the future & Don't affect the future \\
\midrule
\textbf{Reversibility} & Difficult reversibility & Reversible \\
\midrule
\textbf{Impact} & Broad impact & Little impact \\
\midrule
\textbf{Quality Factors} & Affect many important & Affect few important \\
\midrule
\textbf{Periodicity} & Exceptional & Frequent \\
\bottomrule
\end{tabular}
\end{table}

Decisions can also be classified as \textbf{programmed} or \textbf{non-programmed}. Programmed decisions have a well-defined step-by-step sequence that is known and can be followed. For example, in case of an emergency, one calls the emergency number. Non-programmed decisions are unique and specific to the situation, with no defined rules or steps to follow.

\subsection{Problem Classification}
\label{subsec:problem-classification}

Problems can be classified as \textbf{structured} or \textbf{non-structured}:

\begin{itemize}
    \item \textbf{Structured problems}: The problem contains all the information needed to solve it. All necessary data, constraints, and conditions are available from the start.
    \begin{itemize}
        \item \textit{Example}: Solving a system of linear equations where all coefficients and constants are given.
    \end{itemize}
    
    \item \textbf{Non-structured problems}: The problem does not contain all the information needed to solve it. To solve it, we need to search for additional information.
    \begin{itemize}
        \item \textit{Example}: Diagnosing a medical condition where symptoms are present but additional tests, patient history, or expert consultation are required to reach a diagnosis.
    \end{itemize}
\end{itemize}

