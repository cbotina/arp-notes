\subsection{Types of Logic}
\label{subsec:types-logic}

There are several types of logic, all focused on understanding reasoning and determining if it's correct or incorrect. They study statements beyond natural language, extending to mathematics and computer science with very different structures.

\begin{itemize}
    \item \textbf{Propositional logic}: Uses propositions (statements that can be true or false) connected with logical operators ($\land$ [and], $\lor$ [or], $\neg$ [not]) and rules with implication ($\rightarrow$). Uses inference mechanisms like modus ponens and modus tollens.
    
    \item \textbf{Predicate logic}: Extends propositional logic by adding quantifiers:
    \begin{itemize}
        \item $\forall$ (for all)
        \item $\exists$ (there exists)
    \end{itemize}
    Uses traditional inference mechanisms (modus ponens, modus tollens). Example: PROLOG programming language.
    
    \item \textbf{First-order logic}: Formal system for studying inference in first-order languages. Uses quantifiers over individual variables with predicates and functions. Establishes objects and relationships between them. Foundation of computational logic.
    
    \item \textbf{Formal logic} (classical/Aristotelian): Studies propositions and arguments from a structural perspective. Focuses on argument structure, not content truth/falsity. Includes:
    \begin{itemize}
        \item \textbf{Deductive logic}
        \item \textbf{Inductive logic}
    \end{itemize}
    
    \item \textbf{Symbolic/Mathematical logic}: Uses symbols to build a new language for expressing arguments. Translates human thought to mathematical language, converting abstract thinking into formal structures. Used in mathematics to prove theorems.
    
    \item \textbf{Class logic}: Based on set theory. Analyzes logical propositions about membership (or non-membership) of an element to a class (set of elements sharing a characteristic).
    
    \item \textbf{Material logic}: Studied from epistemology. Includes uncertainty—conclusions involve some degree of doubt. Proves validity of reasoning based on reality.
    
    \item \textbf{Natural logic}: Related to empiricism, learning through trial and error. Innate reason that prevents humans from repeating the same mistake.
    
    \item \textbf{Scientific logic}: Extends natural logic by including reason, creating frameworks for everything that exists. Based on finding reasons or justifications for why facts occur.
    
    \item \textbf{Informal logic}: Focuses on language and meaning of semantic constructions and arguments. Differs from formal logic by focusing on sentence content rather than structure.
    
    \item \textbf{Modern logic}: Born in the 19th century, differs from classical logic by including mathematical and symbolic elements, theorems that replace formal logic limitations. Includes:
    \begin{itemize}
        \item \textbf{Modal logic}: Adds modal operators to determine if statements are true/false. Considers expressions like "always", "very likely", "sometimes", "maybe".
        \item \textbf{Mathematical logic}
        \item \textbf{Trivalent logic}
    \end{itemize}
    
    \item \textbf{Computational logic}: Derives from symbolic/mathematical logic (first-order) and applies to computer science. Enables working with programming languages for specific verification tasks.
\end{itemize}

The logics most relevant to artificial intelligence (which will be covered in detail later) are: mathematical logic, description logic (ALC), higher-order logic, multivalued logic, and fuzzy logic.

