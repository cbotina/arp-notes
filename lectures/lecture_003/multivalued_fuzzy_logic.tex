\section{Multivalued Logic and Fuzzy Logic}
\label{sec:multivalued-fuzzy-logic}

\subsection{Multivalued Logic}
\label{subsec:multivalued-logic}

Multivalued logic is a logic that allows intermediate values (large, warm, far, few, many, etc.) and uses more than two truth values to describe concepts that go beyond true and false. Multivalued logics provide conceptual tools that make it possible to formally describe fuzzy, vague, or uncertain information.

\textbf{Key characteristics}:
\begin{itemize}
    \item Uses more than two truth values (unlike classical logic which only has true/false)
    \item Allows for degrees of truth (e.g., 0.0 to 1.0, or linguistic values like "very true", "somewhat true", "false")
    \item Suitable for representing imprecise or uncertain information
\end{itemize}

\textbf{Example}: Instead of saying "The temperature is hot" (true/false), multivalued logic allows us to say "The temperature is 0.8 hot" or "The temperature is somewhat hot", representing degrees of truth.

\subsection{Fuzzy Logic}
\label{subsec:fuzzy-logic}

Fuzzy logic (also called fuzzy logic) is a multivalued logic that allows the mathematical representation of uncertainty and vagueness, providing formal tools for their treatment. Lotfi A. Zadeh is considered the father of fuzzy logic. His career focused on work on fuzzy sets and the application of fuzzy logic in approximate reasoning. The term "fuzzy logic" first appeared in 1974.

\begin{tcolorbox}[colback=blue!5!white,colframe=blue!75!black,title=Curious Fact: Zadeh's Principle]
\textbf{Zadeh's Principle} (1973): "As complexity increases, precise statements lose their meaning and useful statements lose precision." 

This can be summarized as "you can't see the forest for the trees" — when dealing with complex systems, trying to be too precise can make statements meaningless, while useful statements often need to sacrifice some precision to remain comprehensible and applicable.
\end{tcolorbox}

\textbf{Key insight}: The model of characterizing a problem through fuzzy logic is based on the prerogative that the mapping between concepts is done through semantics, not numerical precision. It is very suitable for modeling problems from expert knowledge, which normally details their knowledge base in the form of imprecise expressions.

\textbf{Applications in Artificial Intelligence}:
\begin{itemize}
    \item Handling reasoning under uncertainty and with imprecise notions
    \item Management of databases and knowledge-based systems when information is known to be imprecise
    \item Automation of data mining techniques, which are often linked to fuzzy or multivalued sets
    \item Automatic reasoning methods for these logics
\end{itemize}

\subsection{Example: Air Conditioning System}
\label{subsec:fuzzy-ac-example}

Imagine a fuzzy system controlling an air conditioner that regulates temperature according to needs.

\textbf{Inputs}: The fuzzy chips of the air conditioner collect input data, which in this case could be simply:
\begin{itemize}
    \item Temperature
    \item Humidity
\end{itemize}

\textbf{Processing}: These data are subject to the rules of the inference engine (in the form IF... THEN...), which derives a results area.

\textbf{Example rules}:
\begin{itemize}
    \item IF temperature is \textit{very hot} AND humidity is \textit{high}, THEN cooling is \textit{maximum}
    \item IF temperature is \textit{warm} AND humidity is \textit{moderate}, THEN cooling is \textit{moderate}
    \item IF temperature is \textit{cool} AND humidity is \textit{low}, THEN cooling is \textit{minimum}
\end{itemize}

\textbf{Output}: From the results area, the center of gravity is chosen, providing it as an output. According to the result, the air conditioner could increase or decrease the temperature based on the output degree.

\textbf{Key advantage}: Unlike traditional systems that use precise thresholds (e.g., "if temperature > 25°C, turn on"), fuzzy systems handle gradual transitions and imprecise concepts naturally, making them more suitable for human-like reasoning and expert knowledge representation.


