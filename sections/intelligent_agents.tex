\section{Intelligent Agents}

An \textbf{intelligent agent} is an autonomous entity that perceives its environment through sensors and acts upon that environment through actuators to achieve its goals or objectives. Intelligent agents are characterized by their ability to operate independently, make decisions based on their perceptions, and take actions that affect their environment in pursuit of their goals.

An agent is considered intelligent based on its \textbf{autonomy} and \textbf{rationality}:

\begin{itemize}
    \item \textbf{Autonomy}: The agent operates independently without direct human intervention or control. It has control over its own actions and internal state.
    
    \item \textbf{Rationality}: The agent acts in a way that maximizes its performance measure, given the available information and its knowledge. A rational agent selects actions that are expected to achieve its goals most effectively.
\end{itemize}

\subsection{Phases of Agent Decision Making}

An agent must go through three fundamental phases: \textbf{feel}, \textbf{think}, and \textbf{act}.

\begin{itemize}
    \item \textbf{Feel}: Using perception of the environment through their sensors, the agent extracts and processes information. This phase involves gathering raw data from the environment and converting it into a usable format.
    
    \item \textbf{Think}: The agent reasons and decides through a deliberative process, using the information obtained from the environment and its internal memory. This phase involves analyzing the current situation, considering possible actions, and selecting the best course of action to achieve its goals.
    
    \item \textbf{Act}: Through actuators, the agent produces changes in the environment that will help it achieve its goal. For this, the agent must convert its decisions into information that the actuators can understand and execute.
\end{itemize}

\textit{Example}: Consider a robotic vacuum cleaner agent. In the \textbf{feel} phase, it uses sensors (cameras, bump sensors, dirt detectors) to perceive the room layout, detect obstacles, and identify dirty areas. In the \textbf{think} phase, it processes this information along with its internal map and battery level, deciding which areas to clean next and planning an efficient path. In the \textbf{act} phase, it converts these decisions into motor commands that control its wheels and vacuum mechanism, moving through the room and cleaning the identified areas.

