\section{Problem Solving Stages}

\subsection{First Stage: Understand the Problem's Complexity}

There are many techniques to understand a problem's complexity, which can be organized into two main categories:

\begin{itemize}
    \item \textbf{Identify the problem}: Self-questioning about the problem (origin, magnitude, focus, or history), SWOT analysis (Strengths, Weaknesses, Opportunities, Threats), discussion meetings (Scrum), etc.
    
    \item \textbf{Explain the problem}: Go beyond the surface and investigate the underlying causes of the problem. This can be done using various techniques such as ERIM problem classification, the 20 causes technique, the Ishikawa diagram (fishbone diagram), and other root cause analysis methods.

\end{itemize}

We must, therefore, create problem definitions that present characteristics that allow us to work with them efficiently. Some characteristics or assumptions that can be made in simple environments (well-defined problems) are:

\begin{itemize}
    \item \textbf{Discrete}: The world can be conceived in states. In each state there is a finite set of perceptions and actions.
    
    \item \textbf{Accessible}: The agent can access the relevant characteristics of the environment. It can determine the \textbf{current state} of the world and the \textbf{state it would like to reach}.
    
    \item \textbf{Static and deterministic}: There is no temporal pressure nor uncertainty. The world changes only when the agent acts. The result of each action is totally defined and predictable.
\end{itemize}

\subsection{Second Stage: Create a Strategy}

The steps for creating a strategy are:

\begin{enumerate}
    \item \textbf{Define strategies}: Generate potential strategies using techniques such as brainstorming, 4x4x4, etc.
    
    \item \textbf{Choose a strategy}: Select a strategy by evaluating:
    \begin{itemize}
        \item Benefits
        \item Probability of success
        \item Dependencies
        \item Resources needed (time, cost)
    \end{itemize}
    
    \item \textbf{Design the strategy}: Create a roadmap defining which actions will be performed. This involves planning the sequence and details of the actions to be executed.
\end{enumerate}

\subsection{Third Stage: Solve the Problem}

This stage is achieved by implementing the strategy, evaluating the outcome, and if necessary, refining the strategy.


