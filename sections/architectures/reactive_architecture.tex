\subsection{Reactive Architecture}

A \textbf{reactive architecture} (also known as a \textit{behavior-based} architecture) is characterized by the agent's direct mapping from perceptions to actions without maintaining an internal world model or engaging in complex reasoning. The agent responds quickly to environmental stimuli through simple stimulus-response rules.

\textbf{Key characteristics:}
\begin{itemize}
    \item No internal world model or symbolic representation
    \item Direct mapping from sensors to actuators
    \item Simple stimulus-response rules or behaviors
    \item Fast response time
    \item Emergent behavior from simple rules
\end{itemize}

\textit{Example}: Consider a simple obstacle-avoidance robot with a reactive architecture. The robot has sensors on its front and sides, and it follows these simple rules:
\begin{itemize}
    \item If the front sensor detects an obstacle, turn right
    \item If the right sensor detects an obstacle, turn left
    \item If no obstacles are detected, move forward
    \item If both sensors detect obstacles, move backward
\end{itemize}

The robot doesn't maintain a map of its environment or plan a path. It simply reacts to what it perceives at each moment. This makes it very fast and efficient for simple tasks, though it may not find the optimal path. The robot's navigation behavior emerges from these simple reactive rules.

Figure~\ref{fig:reactive_planner_architecture} illustrates a reactive agent architecture that incorporates a \textbf{Reactive Planner} component:

\begin{itemize}
    \item \textbf{Reactive Planner}: Positioned at the top, this component generates reactive action sequences in direct response to the current state. Unlike deliberative planners, it does not maintain a complex world model but instead quickly generates actions based on immediate perceptions. It receives failure signals from Monitoring and sends actions back to Monitoring.
    \item \textbf{Monitoring}: Positioned in the middle, it monitors the execution of actions and the current state. It receives state information from Execution, sends failure signals to the Reactive Planner when deviations occur, receives actions from the Reactive Planner, and sends action commands to Execution. It also receives external plans.
    \item \textbf{Execution}: Located at the bottom, it interacts directly with the environment through sensors and actuators. It receives sensor data from the environment, updates Monitoring with the current state, receives action commands from Monitoring, and executes actions in the environment.
\end{itemize}


\begin{figure}[H]
    \centering
    \begin{tikzpicture}[
        box/.style={rectangle, draw=black, thick, minimum width=3cm, minimum height=1cm, align=center},
        env/.style={cloud, draw=gray!50, fill=gray!20, minimum width=4cm, minimum height=2cm, cloud puffs=10, cloud puff arc=120},
        arrow/.style={->, >=stealth, dashed, thick},
        label/.style={font=\small}
    ]
    
    % Agent components (enclosed in dashed box)
    \begin{scope}[local bounding box=agent]
        % Reactive Planner (top) - centered
        \node[box] (reactive_planner) at (0,2) {Reactive Planner};
        
        % Monitoring (middle) - centered
        \node[box] (monitoring) at (0,0) {Monitoring};
        
        % Execution (bottom) - centered
        \node[box] (execution) at (0,-2) {Execution};
    \end{scope}
    
    % Draw dashed box around agent (ending above environment)
    \draw[dashed, gray, thick] ([shift={(-2cm,1cm)}]agent.north west) rectangle ([shift={(2cm,-0.5cm)}]agent.south east);
    \node[above] at ([shift={(0,1cm)}]agent.north) {\textbf{Agent}};
    
    % Plan source (outside agent boundaries, to the left of monitoring)
    \node (plan_source) at (-5,0) {plan};
    
    % Environment (outside agent boundaries)
    \node[env] (env) at (0,-6.5) {Environment};
    
    % Arrows
    % Plan to Monitoring
    \draw[arrow] (plan_source.east) -- node[above, label] {} (monitoring.west);
    
    % Environment to Execution (sensor) - UP arrow, goes on the right
    \draw[arrow] ([xshift=5pt]env.north) -- node[right, label] {sensor} ([xshift=5pt]execution.south);
    
    % Execution to Environment (act) - DOWN arrow, goes on the left
    \draw[arrow] ([xshift=-5pt]execution.south) -- node[left, label] {act} ([xshift=-5pt]env.north);
    
    % Execution to Monitoring (state)
    \draw[arrow] ([xshift=5pt]execution.north) -- node[right, label] {state} ([xshift=5pt]monitoring.south);
    
    % Monitoring to Reactive Planner (failure)
    \draw[arrow] ([xshift=5pt]monitoring.north) -- node[right, label] {failure} ([xshift=5pt]reactive_planner.south);
    
    % Reactive Planner to Monitoring (action)
    \draw[arrow] ([xshift=-5pt]reactive_planner.south) -- node[left, label] {action} ([xshift=-5pt]monitoring.north);
    
    % Monitoring to Execution (action)
    \draw[arrow] ([xshift=-5pt]monitoring.south) -- node[left, label] {action} ([xshift=-5pt]execution.north);
    
    \end{tikzpicture}
    \caption{Reactive agent architecture with Reactive Planner}
    \label{fig:reactive_planner_architecture}
    \end{figure}



The flow operates as follows:
\begin{enumerate}
    \item An external \textbf{plan} is provided to the Monitoring component (this could come from a higher-level planner or user input).
    \item The Environment sends \textbf{sensor} data to the Execution component, providing raw information about the current state of the environment.
    \item Execution processes the sensor data and updates Monitoring with the current \textbf{state}.
    \item Monitoring evaluates the state against the plan. If there is a deviation or problem, it sends a \textbf{failure} signal to the Reactive Planner.
    \item The Reactive Planner receives the failure signal and immediately generates a reactive \textbf{action} response, which it sends back to Monitoring.
    \item Monitoring receives the action from the Reactive Planner and forwards the \textbf{action} command to Execution.
    \item Execution \textbf{acts} upon the environment based on the action received from Monitoring.
\end{enumerate}

This architecture emphasizes fast, reactive responses to environmental changes while still allowing for external plan guidance. The Reactive Planner provides quick adaptation without the overhead of maintaining complex internal models. The bidirectional flow between Monitoring and the Reactive Planner enables rapid failure detection and action generation.

\subsubsection{Examples of Reactive Architectures}

\begin{itemize}
    \item \textbf{Subsumption architecture}: A layered architecture where behaviors are organized in levels of increasing complexity. Lower-level behaviors (like obstacle avoidance) can subsume or override higher-level behaviors (like exploration) when triggered by environmental conditions. Each layer operates independently and reactively, with no central control or world model.
    
    \item \textbf{Agent network architecture from Pattie Maes}: A distributed reactive architecture where multiple simple agents (or behaviors) are connected in a network. Each agent has local rules and can activate or inhibit other agents. The overall behavior emerges from the interactions between these agents, without centralized planning.
    
    \item \textbf{Reactive execution model}: Is domain independent and operates with structures precalculated at runtime.
\end{itemize}
